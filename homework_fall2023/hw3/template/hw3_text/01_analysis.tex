\section{Multistep Q-Learning}

\ifsolutions
\input{hw3_solutions/01_solutions}
\fi

\def\solve#1{\csname solution to #1\endcsname}

\begingroup
\def\Q{Q_{\phi_k}}
\def\Qn{Q_{\phi_{k+1}}}
\def\D{\mathcal{D}}

Consider the $N$-step variant of Q-learning described in lecture. We learn $\Qn$ with the following updates:\begin{align}
  y_{j,t} &\gets \biggl(\;\sum_{t'=t}^{t+N-1} \gamma^{t'-t} r_{j,t'}\biggr)+\gamma^{N} \max _{\mathbf{a}_{j,t+N}} \Q\left(\mathbf{s}_{j,t+N}, \mathbf{a}_{j,t+N}\right) \label{eq:q_target}\\
  \phi_{k+1} &\gets \underset{\phi\in\Phi}{\arg\min}  \sum_{j,t} \bigl( y_{j,t}-Q_{\phi}(\mathbf s_{j,t},\mathbf a_{j,t}) \bigr)^2 \label{eq:q_update}
\end{align}
In these equations, $j$ indicates an index in the replay buffer of trajectories $\D_k$. We first roll out a batch of $B$ trajectories to update $\D_k$ and compute the target values in \eqref{eq:q_target}. We then fit $\Qn$ to these target values with \eqref{eq:q_update}. 
After estimating $\Qn$, we can then update the policy through an argmax:\begin{align}
  \pi_{k+1}\left(\mathbf{a}_{t} \mid \mathbf{s}_{t}\right)\gets \left\{\begin{array}{l}1 \text { if } \mathbf{a}_{t}=\arg \max _{\mathbf{a}_{t}} Q_{\phi_{k+1}}\left(\mathbf{s}_{t}, \mathbf{a}_{t}\right) \\ 0 \text{ otherwise. }\end{array}\right. \label{eq:policy_improvement}
\end{align}
We repeat the steps in \cref{eq:q_target,eq:q_update,eq:policy_improvement} $K$ times to improve the policy. In this question, you will analyze some properties of this algorithm, which is summarized in \Cref{alg:multi}.

\begin{algorithm}
\caption{Multistep Q-Learning}
\label{alg:multi}
\begin{algorithmic}[1]
	\Require{iterations $K$, batch size $B$}
	\State initialize random policy $\pi_0$, sample $\phi_0\sim\Phi$
	\For{$k=0\ldots K-1$}
		\State Update $\D_{k+1}$ with $B$ new rollouts from $\pi_k$ \label{eq:data}
		\State compute targets with \eqref{eq:q_target}
		\State $Q_{\phi_{k+1}} \gets$ update with \eqref{eq:q_update}
		\State $\pi_{k+1} \gets$ update with \eqref{eq:policy_improvement}
	\EndFor
	\State\Return $\pi_{K}$
\end{algorithmic}	
\end{algorithm}



\def\makecols#1#2{{\def\p{#2}\newcount\i\i0\hfill\loop\advance\i1\makebox[1cm][c]{\expandafter\p\the\i}\kern.5cm\ifnum\i<#1\repeat\kern-1cm}}
\def\heading#1{\bf\expandafter\uppercase\expandafter{\romannumeral#1.}}
\def\boxes#1{\ensuremath\square}
\def\filled#1#2|#3{\ifnum#1=#3\ensuremath\blacksquare
	\else\if\relax#2\relax\ensuremath\square
	\else\filled#2|#3\fi\fi}
\def\ncol{3}
\newcommand{\checkeditem}[2]{\edef\x{0#1}\item[\expandafter\filled\x|#2]}


\def\choices#1#2{
	\begin{enumerate}
	\item on-policy in tabular setting \makecols\ncol{\filled0#1|}
	\item off-policy in tabular setting \makecols\ncol{\filled0#2|}
	\end{enumerate}
}

\subsection{TD-Learning Bias (2 points)}
\label{q:td_bias}

\def\answer{} % <--- TODO: insert index (0/1) of answer
\ifsolutions\solve\thesubsection\fi
We say an estimator $f_\D$ of $f$ constructed using data $\D$ sampled from process $P$ is \textit{unbiased} when $\mathbb{E}_{\D\sim P}[f_\D(x)-f(x)]=0$ at each $x$.

Assume $\hat Q$ is a noisy (but unbiased) estimate for $Q$. Is the Bellman backup $\mathcal{B}\hat Q = r(s, a) + \gamma \max_{a'} \hat Q(s', a')$ an unbiased estimate of $\mathcal{B}Q$?
\begin{itemize}
    \checkeditem\answer1 Yes
    \checkeditem\answer2 No
\end{itemize}

\textbf{Solution:}

我们知道:

* $\hat Q$ 是无偏的:

  $$
  \mathbb{E}[\hat Q(s,a)] = Q(s,a)
  $$

但是现在我们不是直接用 $\hat Q$,而是用了一个非线性操作:

$$
\max_{a'} \hat Q(s', a')
$$

问题就出在这里。

---

❌ 非线性操作破坏无偏性

“max” 是一个**非线性算子**,因此:

$$
\mathbb{E}[\max_{a'} \hat Q(s', a')] \neq \max_{a'} \mathbb{E}[\hat Q(s', a')]
$$

换句话说,即使每个动作的 $\hat Q(s',a')$ 都是无偏的,
取最大值之后,期望值会 **偏高**(overestimate)。

---

🔎 举个简单例子

假设只有两个动作 $a_1, a_2$,它们的真实值一样:

$$
Q(s',a_1) = Q(s',a_2) = 1
$$

而估计有噪声:

$$
\hat Q(s',a_1) = 1 + \epsilon_1,\quad \hat Q(s',a_2) = 1 + \epsilon_2
$$

其中 $\epsilon_i$ 是零均值噪声(即无偏)。

那我们有:

$$
\mathbb{E}[\max(\hat Q_1, \hat Q_2)] = 1 + \mathbb{E}[\max(\epsilon_1, \epsilon_2)] > 1
$$

所以即使单个估计无偏,取最大值后整体**偏高**。

---

🧩 结论

因此:

$$
\boxed{
\mathcal{B}\hat Q = r + \gamma \max_{a'} \hat Q(s',a')
}
$$

并不是 $\mathcal{B}Q = r + \gamma \max_{a'} Q(s',a')$ 的无偏估计。
也就是说,**Bellman backup 有正偏差(overestimation bias)**。

正确答案是:**No**。

\subsection{Tabular Learning (6 points total)}
\label{q:tabular_learning}

At each iteration of the algorithm above after the update from \cref{eq:q_update}, $\Q$ can be viewed as an estimate of the true optimal $Q^*$. Consider the following statements: 
\begin{enumerate}[label=\bf\Roman*.]
  \item $Q_{\phi_{k+1}}$ is an unbiased estimate of the $Q$ function of the last policy, $Q^{\pi_k}$.
  \item As $k\to\infty$ for some fixed $B$, $\Q$ is an unbiased estimate of $Q^*$, i.e., $\lim_{k\to\infty} \mathbb{E}\bigl[\Q(s,a)-Q^*(s,a)]=0$.
  \item In the limit of infinite iterations and data we recover the optimal $Q^*$, i.e., $\lim_{k,B\to\infty}\mathbb{E}\,\bigl[\|\Q-Q^*\|_\infty\bigr]=0$.
\end{enumerate}

We make the additional assumptions: 
\begin{itemize}
	\item The state and action spaces are finite.
	\item Every batch contains at least one experience for each action taken in each state.
	\item In the tabular setting, $\Q$ can express any function, i.e., $\{\Q:\phi\in\Phi\}=\mathbb{R}^{S\times A}$.
\end{itemize}
When updating the buffer $\D_k$ with $B$ new trajectories in \cref{eq:data} of \Cref{alg:multi}, we say:
\begin{itemize}
    \item When learning \textit{on-policy}, $\D_k$ is set to contain only the set of $B$ new rollouts of $\pi$ (so $\lvert \D_k \rvert = B$). Thus, we only train on rollouts from the current policy.
	\item When learning \textit{off-policy}, we use a fixed dataset $\D_k=\D$ of $B$ trajectories from another policy $\pi'$. 
\end{itemize}

Indicate which of the statements \textbf{I-III} always hold in the following cases. No justification is required.
\ifsolutions\solve\thesubsection\else
\begin{enumerate}
\item $N=1$ and \ldots \makecols\ncol\heading\choices
	{} % <--- TODO: select numbers of boxes to fill; e.g., {13} to select I and III
	{} % <--- TODO: "
\item $N>1$ and \ldots \choices
	{} % <--- TODO: "
	{} % <--- TODO: "
\item In the limit as $N\to\infty$ (no bootstrapping) \ldots \choices
	{} % <--- TODO: "
	{} % <--- TODO: "
\end{enumerate}
\fi

\textbf{Solution:}
好~我们把题目里的三条断言 **I–III**,在 6 个场景($N{=}1$/$N{>}1$/$N{\to}\infty$ × on/off-policy)**逐一**解释清楚。先回顾三条断言含义:

* **I**:一次更新后,$Q_{\phi_{k+1}}$ 是**上一轮策略** $\pi_k$ 的 $Q^{\pi_k}$ 的**无偏估计**。

* **II**:固定批量大小 $B$,当 $k\to\infty$ 时,$Q_{\phi_k}$ 是 $Q^*$ 的**无偏估计**($\lim_{k\to\infty}\mathbb E[Q_{\phi_k}-Q^*]=0$)。

* **III**:当 $k,B\to\infty$(无限迭代且每次有无限数据)时,恢复最优 $Q^*$($\lim_{k,B\to\infty}\mathbb E\|Q_{\phi_k}-Q^*\|_\infty=0$)。

题目假设:**tabular** 表示能力完备;状态动作有限;每个训练 batch 对每个 $(s,a)$ 都有至少一次样本(覆盖性)。

---

预备:几个算子与 N 步目标

* **策略评估算子**:$(\mathcal T^{\pi}Q)(s,a)=r(s,a)+\gamma\,\mathbb E_{s'}\mathbb E_{a'\sim\pi}[Q(s',a')]$

* **最优算子**:$(\mathcal T^*Q)(s,a)=r(s,a)+\gamma\,\mathbb E_{s'}[\max_{a'}Q(s',a')]$(收敛不动点为 $Q^*$)

* **N 步最优 backup**(题里目标):

  $$
  y_{t}=\sum_{i=0}^{N-1}\gamma^i r_{t+i}+\gamma^N\max_{a}Q(s_{t+N},a).
  $$

  on-policy 时,前 $N{-}1$ 步动作来自 $\pi_k$;off-policy 时,来自行为策略 $\pi'$。

要点:只要目标里出现 **$\max$**,它对应的是朝 **最优算子 $\mathcal T^*$** 的更新,而不是对 $\pi_k$ 的评估 $\mathcal T^{\pi_k}$。

在强化学习里,一个策略 π 决定了动作选择方式:

$$
a_t \sim \pi(\cdot|s_t)
$$

● on-policy

* 你在学习 $Q^{\pi}$(当前策略的价值函数)。
* 数据样本也是由同一个策略 π 产生的。

即:

$$
\text{采样策略} = \text{学习目标策略} = \pi
$$

举例:
你用当前的 ε-greedy 策略与环境交互,并用这些数据来更新自己的 Q 值。
→ 这是 on-policy(比如 **SARSA** 算法)。

---

● off-policy

* 你想学习 $Q^{\pi}$(目标策略的价值函数),
  但数据是由另一个策略 π′ 产生的。

即:

$$
\text{采样策略 } \pi' \;\neq\; \text{目标策略 } \pi
$$

举例:
你在看别人玩游戏(π′),但你想学习最优策略 π\*。→ 这是 off-policy(比如 **Q-learning** 算法)。

On-policy 更新(例如 SARSA):

$$
Q(s_t, a_t) \leftarrow Q(s_t, a_t) + \alpha \big[r_t + \gamma Q(s_{t+1}, a_{t+1}) - Q(s_t, a_t)\big]
$$

👉 下一步动作 $a_{t+1}$ 是按照**当前策略 π** 选的。

---

Off-policy 更新(例如 Q-learning):

$$
Q(s_t, a_t) \leftarrow Q(s_t, a_t) + \alpha \big[r_t + \gamma \max_{a'} Q(s_{t+1}, a') - Q(s_t, a_t)\big]
$$

👉 下一步使用 $\max_{a'}$,表示学习的是**最优策略 π\***,
而不是生成数据的行为策略。

---

1) $N=1$

(a) on-policy(每次用 $\pi_k$ 采样 B 条新轨迹,仅用本次数据训练)

* **I:不成立。** 一步目标是 $r+\gamma\max_{a'}Q_k(s',a')$,这是 $\mathcal T^*Q_k$ 的样本近似,估计的是最优策略而不是上一轮的策略,不是 $\mathcal T^{\pi_k}Q_k$。所以并非在无偏评估 $Q^{\pi_k}$。

* **II:不成立。** $B$ 固定、每次只用当轮有限样本,且目标含 $\max$(**maximization bias**)。即便 $k\to\infty$,“无偏估计 $Q^*$” 太强,$\mathbb{E}[\max_{a'} \hat Q(s', a')] \neq \max_{a'} \mathbb{E}[\hat Q(s', a')]$达不到。

* **III:成立。** 在 tabular + 完全覆盖 + $B\to\infty$ 的极限下,每次 backup 逼近 $\mathcal T^*$,而 $\mathcal T^*$ 在 $\|\cdot\|_\infty$ 下是 $\gamma$-收缩映射,迭代收敛到唯一不动点 $Q^*$。

(b) off-policy(固定一个来自 $\pi'$ 的数据集 $\mathcal D$)

* **I:不成立。** 同上,目标是朝 $\mathcal T^*$ 而非 $\mathcal T^{\pi_k}$;且数据分布与 $\pi_k$ 不同,更不满足“无偏评估 $Q^{\pi_k}$”。

* **II:不成立。** 理由同上。

* **III:不成立。** 如果数据对每个 $(s,a)$ 有覆盖,**Fitted Q-Iteration(FQI)**:$B\to\infty$ 时经验 backup $\to \mathcal T^*$,重复迭代收敛到 $Q^*$(与行为策略无关,因为一步 backup 中并不需要修正分布)。$y_t = r_t + \gamma \max_{a'} Q(s_{t+1}, a')$这个不包含任何来自行为策略 π′ 的动作选择,即时奖励$r_{t}$只由当前动作状态决定和下一个状态$s_{t+1}$是由分布环境$P(s_{t+1}|s_t,a_t)$决定的。但是此处的:Learning off-policy with fixed dataset $\mathcal D$ of $B$ trajectories from another policy $\pi'$,如果数据集$\mathcal D$中没有覆盖所有的$(s,a)$对,那么就无法保证收敛到$Q^*$。因此III不成立。

2) $N>1$(有限多步)

(a) on-policy

* **I:不成立。** 目标:前 $N$ 步是真实回报,但**第 $N$ 步用 $\max$ 自举**,它并非 $\mathcal T^{\pi_k}$(后者应在每一步都对 $\pi_k$ 取期望)。

* **II:不成立。** 依旧是有限样本 + $\max$ 带来的偏差,固定 $B$ 不能保证“无偏到 $Q^*$”。

* **III:成立。** 在 tabular、覆盖且 $B\to\infty$ 的极限下,N 步最优 backup 的期望等价于把 $\mathcal T^*$ 置于第 $N$ 步(可视作 $\gamma^N$-收缩),配合贪心改进形成广义策略迭代,最终收敛到 $Q^*$。

(b) off-policy

* **I:不成立。** 前 $N{-}1$ 步的奖励来自行为策略 $\pi'$ 的轨迹,然而目标要“朝最优”,分布不匹配且无修正,**不是**对 $Q^{\pi_k}$ 的无偏评估。

* **II:不成立。** 同上。

* **III:不成立。** 多步 off-policy **如果不做重要性采样校正**,即使数据无限,也会收敛到错误的固定点(N 步回报混入了 $\pi'$ 的决策),不能保证得到 $Q^*$。

当 $N>1$ 时,
\[
y_t^{(N)} = r_t + \gamma r_{t+1} + \cdots + \gamma^{N-1} r_{t+N-1}
+ \gamma^N \max_a Q(s_{t+N}, a)
\]

\noindent 问题:这些 $r_{t+i}$ 来自谁?

\begin{quote}
👉 来自行为策略 $\pi'$ 在环境中“连续选择的动作”所产生的轨迹。
\end{quote}

于是:

\begin{itemize}
  \item 奖励序列的统计特性取决于 $\pi'$;
  \item 也就是说,
  \[
  \mathbb{E}_{\pi'}[\,r_t + \gamma r_{t+1} + \dots + \gamma^{N-1} r_{t+N-1}\,]
  \;\neq\;
  \mathbb{E}_{\pi^*}[\cdots].
  \]
\end{itemize}

\noindent 换句话说:

\begin{quote}
你用 $\pi'$ 的行为轨迹去估计 $\pi^*$ 的未来回报,期望错了。
\end{quote}

这导致多步目标不再是 $\mathcal{T}^*$ 的期望,而是某个混合算子:

\[
\tilde{\mathcal{T}}_{N}^{\pi'}(Q)
= \mathbb{E}_{\pi'}\!\left[
  \sum_{i=0}^{N-1}\gamma^i r_{t+i}
  + \gamma^N \max_a Q(s_{t+N}, a)
\right]
\]

\noindent 这个算子的固定点不再是 $Q^*$。

---

3) $N\to\infty$(纯 Monte Carlo,无自举)

(a) on-policy

* **I:成立。** 目标就是完整回报 $G_t$;on-policy 下 $\mathbb E[G_t|s,a]=Q^{\pi_k}(s,a)$,MC 评估对 $Q^{\pi_k}$ **无偏**。

* **II:不成立。** 题目要求“固定 $B$”,每轮只用 $B$ 条新样本做评估并立刻贪心改进,带噪声的改进过程不保证“$\lim_{k\to\infty}$ 无偏到 $Q^*$”;这个断言比“收敛”更强,一般不成立。

* **III:成立。** $B\to\infty$ 时,MC 对 $Q^{\pi_k}$ 的估计一致;与贪心改进交替(MC Policy Iteration)在 tabular 下收敛到 $Q^*$。

(b) off-policy

* **I:不成立。** MC **off-policy** 若不做重要性采样,$\mathbb E_{\pi'}[G_t]\neq Q^{\pi_k}$,评估有系统偏差。想得到$Q^{\pi_k}$,必须用但是采样是在$\pi'$下进行的轨迹,需要用重要性采样校正。

* **II:不成立。** 同理上,没有重要性采样校正,MC off-policy 评估有偏差,不能保证无偏到 $Q^*$。

* **III:不成立。** 无 IS 校正时,MC off-policy 不能保证一致性,更谈不上收敛到 $Q^*$。即便 $B \to \infty$、$k \to \infty$,若不做 IS/校正,MC off-policy 的期望指向的不是$\mathbb{E}_{\pi}[\cdot]$,而是 $\mathbb{E}_{\pi'}[\cdot]$。于是它收敛到错误的固定点(反映 $\pi'$ 轨迹的长期回报),不是 $Q^*$,因此 III 不成立。

\subsection{Variance of $Q$ Estimate (2 points)}
\label{q:variance_estimate}
Which of the three cases ($N = 1$, $N > 1$, $N \to \infty$) would you expect to have the highest-variance estimate of $Q$ for fixed dataset size $B$ in the limit of infinite iterations $k$? Lowest-variance?

\def\highest{} % <--- TODO: insert index of highest variance answer
\def\lowest{} % <--- TODO: insert index of lowest variance answer
\ifsolutions\solve\thesubsection\fi
\begin{minipage}{0.49\linewidth}
Highest variance:\smallskip
\begin{itemize}\itemsep=1ex
    \checkeditem\highest1 $N = 1$
    \checkeditem\highest2 $N > 1$
    \checkeditem\highest3 $N \to \infty$
\end{itemize}
\end{minipage}
\begin{minipage}{0.49\linewidth}
Lowest variance:\smallskip
\begin{itemize}\itemsep=1ex
    \checkeditem\lowest1 $N = 1$
    \checkeditem\lowest2 $N > 1$
    \checkeditem\lowest3 $N \to \infty$
\end{itemize}
\end{minipage}


\textbf{Solution}
**答案:**

* **Highest variance(方差最高):** $N \to \infty$

* **Lowest variance(方差最低):** $N = 1$

**理由(简述):**
$N=1$(一步 TD/Q-learning)在目标中立即**自举**,只用当前奖励 $r_t$ 加上 $\gamma \max_a Q(s_{t+1},a)$ 的估计,因而目标的随机成分最少,**方差最低**(但偏差较大)。
$N\to\infty$(纯 MC)把整条未来回报 $\sum_{i=0}^{T-t-1}\gamma^i r_{t+i}$ 都当作目标,累积了所有随机性(转移、奖励、轨迹长度等),**方差最高**(但无自举偏差)。
$N>1$ 落在两者之间,方差介于二者之间。固定批量 $B$ 时,即使 $k\to\infty$,每次目标的采样噪声不会消失,因此这种方差排序成立

\subsection{Function Approximation (2 points)}
\label{q:function_approximation}
Now say we want to represent $Q$ via function approximation rather than with a tabular representation. Assume that for any deterministic policy $\pi$ (including the optimal policy $\pi^*$), function approximation can represent the true $Q^\pi$ exactly.
Which of the following statements are true?

\def\answer{} % <--- TODO: insert index of answer(s)
\ifsolutions\solve\thesubsection\fi
\begin{itemize}
    \checkeditem\answer1 When $N = 1$, $Q_{\phi_{k+1}}$ is an unbiased estimate of the $Q$-function of the last policy $Q^{\pi_k}$.
    \checkeditem\answer2 When $N = 1$ and in the limit as $B\to\infty,\,k \to \infty$, $\Q$ converges to $Q^*$.
    \checkeditem\answer3 When $N > 1$ (but finite) and in the limit as $B\to\infty,\,k \to \infty$, $\Q$ converges to $Q^*$.
    \checkeditem\answer4 When $N \to \infty$ and in the limit as $B \to \infty,\,k \to \infty$, $\Q$ converges to $Q^*$.
\end{itemize}

\textbf{Solution:}

现在我们希望通过**函数逼近(function approximation)**来表示 $Q$,
而不是使用**表格型表示(tabular representation)**。

假设:
对于任意一个确定性策略 $\pi$(包括最优策略 $\pi^*$),
函数逼近器都可以**精确地表示该策略的 Q 函数 $Q^\pi$**。

问:以下哪些陈述是正确的?

---

**表格型表示(tabular representation)是啥?**

* “Tabular” = “表格式”。

* 表示你直接为每个状态–动作对 $(s,a)$ 存一个值。
  即:

  $$
  Q(s,a) \text{ 被存成一个表格 } \mathbb{R}^{|S|\times|A|}
  $$

  状态空间有限,动作空间有限。

* “Function approximation” 则是:

  $$
  Q_\phi(s,a) = f_\phi(s,a)
  $$

  用参数(如神经网络)表示函数,而不是直接存表。

✅ 选项 1

> 当 $N=1$ 时,$Q_{\phi_{k+1}}$ 是上一次策略的 $Q^{\pi_k}$ 的无偏估计。

👉 这是 **错误的(❌)**。
因为 $N=1$ 的更新(Q-learning)使用的是

$$
y_t = r_t + \gamma \max_{a'} Q(s_{t+1},a')
$$

这对应 **Bellman 最优算子 $\mathcal T^*$**,
不是 $\mathcal T^{\pi_k}$(策略评估算子)。

换句话说:

* 目标是改进策略(learn $Q^*$),
* 不是评估上一个策略,
* 所以它对 $Q^{\pi_k}$ 来说是有偏的。

✅ **结论:不成立。**

---

✅ 选项 2

> 当 $N=1$,且当 $B\to\infty, k\to\infty$ 时,$Q_{\phi_k}$ 收敛到 $Q^*$。

→ **正确(✅)**

解释:

* $N=1$ 表示标准 Q-learning;
* 有无限数据(B→∞)与无限迭代(k→∞);
* 又假设函数逼近可以精确表达真实 Q;
* 因为 $\mathcal T^*$ 是 γ 收缩映射;
* 所以理论上 $Q_{\phi_k}\to Q^*$。

✅ **结论:成立。**

---

✅ 选项 3

> 当 $N>1$(但有限),且 $B\to\infty, k\to\infty$ 时,$Q_{\phi_k}$ 收敛到 $Q^*$。

→ **错误(❌)**

解释:

* 对于多步 $N>1$ 的情况,
  前 N−1 步用真实样本奖励,最后一步自举;
* 这意味着更新对应的算子不再是严格的 $\mathcal T^*$;
* 而是一个“混合”算子,通常会带有 **bootstrapping bias**;
* 即使有无限数据,也不保证完全无偏。

✅ **结论:不成立。**

---

✅ 选项 4

> 当 $N\to\infty$,且 $B\to\infty, k\to\infty$ 时,$Q_{\phi_k}$ 收敛到 $Q^*$。

→ **正确(✅)**

解释:

* $N\to\infty$ 表示完全展开回报(Monte Carlo);
* 如果是 on-policy;
* 则回报期望 = 策略真实回报;
* 每次更新后又做贪心改进(policy improvement);
* 所以和 Monte Carlo Policy Iteration 等价;
* 在无限数据与完美逼近下,收敛到最优 $Q^*$。

✅ **结论:成立。**

\subsection{Multistep Importance Sampling (5 points)}
\label{q:importance_sampling}

We can use importance sampling to make the $N$-step update work off-policy with trajectories drawn from an arbitrary policy. Rewrite \eqref{eq:q_update} to correctly approximate a $\Q$ that improves upon $\pi$ when it is trained on data $\D$ consisting of $B$ rollouts of some other policy $\pi'(\mathbf a_t\mid\mathbf s_t)$. 

Do we need to change \eqref{eq:q_update} when $N=1$? What about as $N\to\infty$? 

You may assume that $\pi'$ always assigns positive mass to each action. [Hint: re-weight each term in the sum using a ratio of likelihoods from the policies $\pi$ and $\pi'$.]

 
\ifsolutions\solve\thesubsection\else
% TODO: answer question above
\fi

\endgroup

\textbf{Solution:}

设**目标策略**为 $\pi$(我们希望学习/改进它),
**行为策略**为 $\pi'$(数据来自它),并令

$$
\rho_{t}\;\triangleq\;\frac{\pi(a_t\mid s_t)}{\pi'(a_t\mid s_t)}\;>0\qquad(\text{题目已给正概率假设})
$$

对每条轨迹的任意起点 $t$,**N 步 off-policy 目标(带逐步重要性采样,per–decision IS)** 为:

$$
\boxed{
y^{(N)}_{t,\text{IS}}
=
\sum_{i=0}^{N-1}\Bigg(\;\gamma^i \prod_{j=0}^{i-1}\rho_{t+j}\Bigg)\, r_{t+i}
\;+\;
\gamma^N \Bigg(\prod_{j=0}^{N-1}\rho_{t+j}\Bigg)\,
\max_{a}Q(s_{t+N},a)
}
$$

(约定空乘积 $\prod_{j=0}^{-1}(\cdot)=1$。若在 $t+N$ 前终止,最后一项去掉/视为 0。)

然后把 $\,y^{(N)}_{t,\text{IS}}$ 代入原来的回归式(eq. (2)):

$$
\phi_{k+1} \;=\; \arg\min_{\phi\in\Phi}\sum_{(j,t)}
\Big(y^{(N)}_{j,t,\text{IS}}-Q_\phi(s_{j,t},a_{j,t})\Big)^2.
$$

> 直观:前 $N$ 步的每一段回报都用对应长度的**权重乘积** $\prod\rho$ 进行校正,使样本在期望上从 $\pi'$ 的分布“重权”成 $\pi$ 的分布;第 $N$ 步的自举项同样乘上长度为 $N$ 的乘积。

---

N=1 / N→∞ 是否需要改?

1) **N=1:不需要改**(off-policy 安全)

$$
y^{(1)}_{t}=r_t+\gamma \max_{a} Q(s_{t+1},a)
$$

这一步目标与 $\pi'$ 的下一步动作无关(直接取 $\max$),因此**无需 IS 权重**;这就是一步 Q-learning 能 off-policy 的原因。

2) **$N\to\infty$**(纯 Monte Carlo):**需要改**

此时目标是整条回报:

$$
y^{(\infty)}_{t,\text{IS}}
=\sum_{i=0}^{T-t-1}\Bigg(\gamma^i\prod_{j=0}^{i-1}\rho_{t+j}\Bigg)\,r_{t+i}
$$

(无自举项)。
没有这些 $\prod\rho$ 的话就是在 $\pi'$ 上求期望,**偏离目标策略 $\pi$**;加入 IS 后才在期望上等于 $\mathbb{E}_\pi[G_t]$。注意这会显著**增大方差**。



