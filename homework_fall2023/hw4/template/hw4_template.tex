\documentclass[11pt]{article}
\usepackage[letterpaper,margin=1in]{geometry}
\usepackage{amsmath,amssymb}
\usepackage{hyperref}
\usepackage{enumitem}
\usepackage{parskip}
\usepackage{bm}
\usepackage{titlesec}
\usepackage{graphicx}
\usepackage{algorithm}
\usepackage[noend]{algpseudocode}  % 提供 \begin{algorithmic}
\usepackage{subcaption} 
\usepackage{float}



% --------------------
\usepackage{fontspec}       % 字体设置
\usepackage{luatexja}       % 中文支持
\usepackage{luatexja-fontspec}
\setmainjfont{FangSong}  % 简体中文字体

% --- math helpers ---
\newcommand{\E}{\mathbb{E}}        % expectation
\newcommand{\Var}{\mathrm{Var}}    % variance (如果用得到)
\newcommand{\R}{\mathbb{R}}        % real numbers
% ===== macros used in the text =====
\newcommand{\Sset}{\mathcal{S}}
\newcommand{\Aset}{\mathcal{A}}
\newcommand{\M}{\mathcal{M}}
\newcommand{\hP}{\widehat{P}}

\newcommand{\Vpi}{V^{\pi}}
\newcommand{\Qpi}{Q^{\pi}}
\newcommand{\Ppi}{P^{\pi}}
\newcommand{\hVpi}{\widehat{V}^{\pi}}
\newcommand{\hQpi}{\widehat{Q}^{\pi}}
\newcommand{\st}{\mathbf{s}_t}
\newcommand{\stp}{\mathbf{s}_{t+1}}
\newcommand{\at}{\mathbf{a}_t}
\newcommand{\f}{f_\theta}
\newcommand{\D}{\mathcal{D}}


\titleformat{\section}{\large\bfseries}{\thesection}{1em}{}
\titleformat{\subsection}{\normalsize\bfseries}{\thesubsection}{1em}{}

\title{Berkeley CS 285: Deep Reinforcement Learning, Decision Making, and Control\\[4pt]
Fall 2023\\[4pt]Assignment 4 -- Model-Based RL}
\author{}
\date{Due September 11, 11:59 pm}

\begin{document}
\maketitle

\section{Introduction}
The goal of this assignment is to get experience with model-based reinforcement learning. In general, model-
based RL consists of two main parts: learning a dynamics function to model observed state transitions, and
then using predictions from that model in some way to decide what to do (e.g., use model predictions to learn
a policy, or use model predictions directly in an optimization setup to maximize predicted rewards).

In this assignment, you will get both theoretical and practical experience with model-based RL. In the analysis
section, you will analyze the effectiveness of a simple count-based model. Before doing that section, it will be
greatly beneficial to first go over lecture 17 of this course on basics of RL theory (if you wish to complete this
section before the lecture, the same lecture from past years will be sufficient); another beneficial resource will
be section 2 of this textbook. Then, in the coding section, you will implement both the process of learning a
dynamics model, as well as the process of creating a controller to perform action selection through the use of
these model predictions. For references to this type of approach, see this paper and this paper.

\section{Analysis}
\paragraph{Setting.}
We have a discounted tabular MDP$M = (\Sset,\Aset,P,r,\gamma)$where $\Sset,\Aset$ are a finite set of states and actions, $P$ is the dynamics model (where $P(\cdot \mid s,a)$ is a probability distribution over states), $r$ is a reward function (where rewards are between $[0,1]$), and $\gamma \in (0,1)$ is the discount factor.

\paragraph{Learning a dynamics model.}
We consider the most naive model-based algorithm. Suppose we have access to a simulator of the environment, and at each state--action pair $(s,a)$, we call our simulator $N$ times retrieving samples $s' \sim P(\cdot \mid s,a)$. Then, we build a dynamics model of the environment as simply:
\[
\widehat{P}(s' \mid s,a) \;=\; \frac{\mathrm{count}(s,a,s')}{N},
\]
where $\mathrm{count}(s,a,s')$ is the number of times we observed $(s,a)$ transitioning to $s'$. For tabular MDPs, we can view $\widehat{P}$ as a matrix of size $\lvert \Sset\rvert\lvert \Aset\rvert \times \lvert \Sset\rvert$.

\paragraph{Additional notation.}
Let $\widehat{\M}$ be an MDP identical to $\M$, except where the true dynamics $P$ is replaced by model $\widehat{P}$. Let $\widehat{V}^{\pi}$, $\widehat{Q}^{\pi}$, $V^{*}$, and $Q^{*}$ denote the value function and state--action value function, and optimal value and state--action value function, in $\widehat{\M}$, respectively.

\paragraph{Problem 2.1.}
In lecture 17, we saw a proof of a lemma called the \emph{Simulation Lemma}, which states that for any policy $\pi$:
\[
\Qpi - \hQpi \;=\; \gamma\,(I-\gamma \Ppi)^{-1} (P-\hP)\,\Vpi.
\]
Prove the following similar lemma, which we dub the \emph{Alternative Simulation Lemma}:
\[
\text{(Alternative Simulation Lemma) }\quad
\forall \pi,\qquad
\Qpi - \hQpi \;=\; \gamma\,(I-\gamma \Ppi)^{-1} (P-\hP)\,\hVpi.
\]

\textbf{Solution}
By policy evaluation in linear-operator form,
\[
Q^\pi=(I-\gamma P^\pi)^{-1}r,\qquad
\hat Q^\pi=(I-\gamma \hat P^\pi)^{-1}r,
\]
hence $r=(I-\gamma \hat P^\pi)\hat Q^\pi$.  Therefore,
\begin{align*}
Q^\pi-\hat Q^\pi
&= (I-\gamma P^\pi)^{-1}r-\hat Q^\pi \\
&= (I-\gamma P^\pi)^{-1}(I-\gamma \hat P^\pi)\hat Q^\pi-\hat Q^\pi \\
&=\Big[(I-\gamma P^\pi)^{-1}(I-\gamma \hat P^\pi)-I\Big]\hat Q^\pi \\
&=(I-\gamma P^\pi)^{-1}\Big[(I-\gamma \hat P^\pi)-(I-\gamma P^\pi)\Big]\hat Q^\pi \\
&=\gamma\,(I-\gamma P^\pi)^{-1}\,(P^\pi-\hat P^\pi)\,\hat Q^\pi \\
&=\gamma\,(I-\gamma P^\pi)^{-1}\,(P-\hat P)\,\Pi\,\hat Q^\pi \\
&=\gamma\,(I-\gamma P^\pi)^{-1}\,(P-\hat P)\,\hat V^\pi ,
\end{align*}
where we used $P^\pi=P\Pi$, $\hat P^\pi=\hat P\Pi$, and $\hat V^\pi=\Pi\hat Q^\pi$.
This proves the claim.

\paragraph{Problem 2.2.}
In lecture 17, we saw how to bound $\lVert \Qpi-\hQpi\rVert_{\infty}$ using the Simulation Lemma and standard concentration arguments. We will attempt to do the same with the Alternative Simulation Lemma derived in Problem 2.1. Which of the following statements (there may be multiple) are correct?

\emph{Hint 1:} A statement is correct if the inequalities referenced are applied correctly, and if their assumptions hold before applying them.

\emph{Hint 2:} For each observed transition from $(s,a)$ to $s'$, you can define a random variable
$X = \mathbf{1}_{s'} \cdot V$ that is the dot product between $\mathbf{1}_{s'} \in \mathbb{R}^{\lvert \Sset\rvert}$ an indicator vector at $s'$ and vector $V \in \mathbb{R}^{\lvert \Sset\rvert}$, and whose expected value is $\mathbb{E}[X] = P(\cdot \mid s,a)\cdot V$. Then, can Hoeffding's inequality be applied to all $N$ observed transitions from $(s,a)$ in this way? Can Hoeffding's inequality be applied for any vector $V$?

\begin{enumerate}[leftmargin=2em,label=\arabic*.]
\item For any policy $\pi$ and $\delta>0$, the following holds with probability at least $1-\delta$,
\begin{align*}
\lVert (P-\hP)\,\hVpi \rVert_{\infty}
&\le \max_{s,a} \lVert P(\cdot\mid s,a)-\hP(\cdot\mid s,a)\rVert_{1}\,\lVert \hVpi\rVert_{\infty} \\
&\le \frac{1}{1-\gamma}\sqrt{\frac{4\lvert \Sset\rvert \log\!\big(\lvert \Sset\rvert \lvert \Aset\rvert/\delta\big)}{N}},
\end{align*}
where we use Hoeffding's inequality and the union bound in the second inequality.
\item For any policy $\pi$ and $\delta>0$, the following holds with probability at least $1-\delta$,
\[
\lVert (P-\hP)\,\hVpi \rVert_{\infty}
\le \frac{1}{1-\gamma}\sqrt{\frac{2\log\!\big(2\lvert \Sset\rvert \lvert \Aset\rvert/\delta\big)}{N}},
\]
using Hoeffding's inequality and the union bound.
\item For $\delta>0$, the following holds with probability at least $1-\delta$,
\[
\lVert (P-\hP)\,V^{*} \rVert_{\infty}
\le \frac{1}{1-\gamma}\sqrt{\frac{2\log\!\big(2\lvert \Sset\rvert \lvert \Aset\rvert/\delta\big)}{N}},
\]
where the inequality arises from Hoeffding's inequality and the union bound.
\item For $\delta>0$, the following holds with probability at least $1-\delta$,
\[
\lVert (P-\hP)\,\hVpi \rVert_{\infty}
\le \frac{1}{1-\gamma}\sqrt{\frac{2\log\!\big(2\lvert \Sset\rvert \lvert \Aset\rvert/\delta\big)}{N}},
\]
where we use Hoeffding's inequality and the union bound.
\end{enumerate}

\emph{Hint 3:} Some people have expressed confusion on how to use Hint 2 in Problem 2.2. Since it wasn't explicitly covered in lecture 17, here is a usage of Hint 2. You can follow the same format where $V$ is replaced with the appropriate vector, and verify if the argument still holds.

Let $V \in \mathbb{R}^{\lvert \Sset\rvert}$ be a vector that does not depend on the observed transitions at all. For each observed transition $i$ that is from $(s,a)$ to $s'_i$, we define the random variable $X_i = \mathbf{1}_{s'_i}\cdot V$ that is the dot product between $\mathbf{1}_{s'_i} \in \mathbb{R}^{\lvert \Sset\rvert}$ an indicator vector at $s'_i$ and vector $V$. We see that $\mathbb{E}[X_i] = P(\cdot \mid s,a)\cdot V$ and that $0 \le X_i \le \lVert V\rVert_\infty$. Note that because $V$ does not depend on the observed data, all $X_i$ are independent. Note also that
\[
\frac{1}{N}\sum_{i=1}^{N} X_i = \hP(\cdot\mid s,a)\cdot V.
\]
Therefore, we can use Hoeffding's inequality to conclude that for any $(s,a)$, and with probability at least $1 - \delta/(\lvert \Sset\rvert \lvert \Aset\rvert)$, we have
\[
\big(P(\cdot\mid s,a)-\hP(\cdot\mid s,a)\big)\cdot V
\;\le\;
\sqrt{\frac{2\lVert V\rVert_\infty^{2}\,\log\!\big(2\lvert \Sset\rvert \lvert \Aset\rvert/\delta\big)}{N}}\,.
\]

\textbf{Solution}
\paragraph{构造与期望.}
对固定的 $(s,a)$,令
\[
s'_i \sim P(\cdot\,|\,s,a),\qquad
X_i := V^\pi(s'_i),
\]
则
\[
\mathbb{E}[X_i]
= \sum_{s'} P(s'|s,a)\,V^\pi(s')
= P(\cdot|s,a)^\top V^\pi .
\]
经验均值可写成
\[
\frac{1}{N}\sum_{i=1}^N X_i
= \sum_{s'} \Big(\tfrac{1}{N}\sum_{i=1}^N \mathbf 1\{s'_i=s'\}\Big) V^\pi(s')
= \hat P(\cdot|s,a)^\top V^\pi .
\]

\paragraph{Hoeffding 不等式.}
设 $X_1,\ldots,X_n$ 相互独立且 $a_i\le X_i\le b_i$,
记 $S_n=\sum_{i=1}^n X_i$,则对任意 $t>0$,
\begin{align*}
\Pr\!\big(S_n-\mathbb{E}[S_n]\ge t\big)
&\le \exp\!\left(-\frac{2t^2}{\sum_{i=1}^n (b_i-a_i)^2}\right),\\
\Pr\!\big(|S_n-\mathbb{E}[S_n]|\ge t\big)
&\le 2\exp\!\left(-\frac{2t^2}{\sum_{i=1}^n (b_i-a_i)^2}\right).
\end{align*}

\paragraph{在本设定中的应用.}
由于 $X_i=V^\pi(s'_i)$,有
\[
X_i \in \big[-\|V^\pi\|_\infty,\ \|V^\pi\|_\infty\big]
\quad\Rightarrow\quad
b_i-a_i \le 2\|V^\pi\|_\infty .
\]
取 $S_N=\sum_{i=1}^N X_i$、$t=N\varepsilon$,两侧界给出
\[
\Pr\!\left(
\left|\frac{1}{N}\sum_{i=1}^N X_i - \mathbb{E}X_i\right|\ge \varepsilon
\right)
\le
2\exp\!\left(-\frac{2N\varepsilon^2}{(2\|V^\pi\|_\infty)^2}\right)
=
2\exp\!\left(-\frac{N\varepsilon^2}{2\|V^\pi\|_\infty^2}\right).
\]
等价地,令右侧为 $\delta$,则以概率至少 $1-\delta$,
\[
\big|(\hat P(\cdot|s,a)-P(\cdot|s,a))^\top V^\pi\big|
\;\le\;
\|V^\pi\|_\infty \sqrt{\frac{2\log(2/\delta)}{N}}\, .
\]

由hint3可知,选项2正确,选项1、3、4均错误。

\section{Model-Based Reinforcement Learning}
We briefly review the specific MBRL pipeline used in this homework. MBRL consists of
(1) learning a dynamics model and (2) using the learned model to plan and execute actions
that minimize a known cost (or maximize a reward).

\subsection{Dynamics Model}
In this assignment, you will learn a neural network dynamics model $f_\theta$ of the form:
\begin{equation}
    \hat{\bm{\Delta}}_{t+1} \;=\; \f(\st,\at).
    \label{eq:delta}
\end{equation}
which predicts the change in state given the current state and action. So given the prediction $\hat{\bm{\Delta}}_{t+1}$, you can
generate the next prediction with
\begin{equation}
    \hat{\mathbf{s}}_{t+1} \;=\; \st + \hat{\bm{\Delta}}_{t+1}.
    \label{eq:next-state}
\end{equation}
See the previously referenced paper for intuition on why we might want our network to predict state differences,
instead of directly predicting next state.

You will train $f_\theta$ in a standard supervised learning setup, by performing gradient descent on the following
objective
\begin{align}
    \mathcal{L}(\theta)
    &= \sum_{(\st,\at,\stp)\in \D}
       \left\|(\stp-\st)-\f(\st,\at)\right\|_2^2
       \label{eq:loss-raw} \tag{3}\\
    &= \sum_{(\st,\at,\stp)\in \D}
       \big\|\bm{\Delta}_{t+1}-\hat{\bm{\Delta}}_{t+1}\big\|_2^2,
       \tag{4}
\end{align}
where $\bm{\Delta}_{t+1} := \stp-\st$.
In practice, it’s helpful to normalize the target of a neural network. So in the code, we’ll train the network to
predict a normalized version of the change in state, as in
\begin{equation}
    \mathcal{L}(\theta)
    = \sum_{(\st,\at,\stp)\in \D}
      \Big\|\mathrm{Normalize}(\stp-\st)-\f(\st,\at)\Big\|_2^2.
    \label{eq:loss-norm} \tag{5}
\end{equation}
Since $f_\theta$ is trained to predict the normalized state difference, you generate the next prediction with
\begin{equation}
    \hat{\mathbf{s}}_{t+1}
    \;=\; \st + \mathrm{Unnormalize}\!\big(\f(\st,\at)\big).
    \label{eq:denorm} \tag{6}
\end{equation}

\subsection{Action Selection}
Given the learned dynamics model, we now want to select and execute actions that minimize a known cost
function (or maximize a known reward function). Ideally, you would calculate these actions by solving the
following optimization:
\begin{equation}
    \mathbf{a}_t^{\star} =
    \arg\min_{\mathbf{a}_{t:\infty}}
    \sum_{t'=t}^{\infty} c(\hat{\mathbf{s}}_{t'},\mathbf{a}_{t'})
    \quad \text{s.t. } \;
    \hat{\mathbf{s}}_{t'+1}=\hat{\mathbf{s}}_{t'}+\f(\hat{\mathbf{s}}_{t'},\mathbf{a}_{t'}).
    \label{eq:inf-opt} \tag{7}
\end{equation}
However, solving Eqn. 7 is impractical for two reasons: (1) planning over an infinite sequence of actions is
impossible and (2) the learned dynamics model is imperfect, so using it to plan in such an open-loop manner
will lead to accumulating errors over time and planning far into the future will become very inaccurate.

Instead, one alternative is to solve the following gradient-free optimization problem:
\begin{equation}
    A^\star = \arg\min_{\{A^{(0)},\ldots,A^{(K-1)}\}}
    \sum_{t'=t}^{t+H-1} c(\hat{\mathbf{s}}_{t'},\mathbf{a}_{t'})
    \quad \text{s.t. } \;
    \hat{\mathbf{s}}_{t'+1}=\hat{\mathbf{s}}_{t'}+\f(\hat{\mathbf{s}}_{t'},\mathbf{a}_{t'}).
    \label{eq:shooting} \tag{8}
\end{equation}

in which $A^{(k)} = (a^{(k)}_t , . . . , a^{(k)}_{t+H−1})$ are each a random action sequence of length H. What Eqn. 8 says is
to consider K random action sequences of length H, predict the result (i.e., future states) of taking each of
these action sequences using the learned dynamics model $f_\theta$, evaluate the cost/reward associated with each
candidate action sequence, and select the best action sequence. Note that this approach only plans H steps
into the future, which is desirable because it prevent accumulating model error, but is also limited because it
may not be sufficient for solving long-horizon tasks.

A better alternative to this random-shooting optimization approach is the cross-entropy method (CEM), which
is similar to random-shooting, but with iterative improvement of the distribution of actions that are sampled
from. We first randomly initialize a set of K action sequences $A^{(0)}, ..., A^{(K−1)}$, like in random-shooting.
Then, we choose the J sequences with the highest predicted sum of discounted rewards as the ”elite” action
sequences. We then fit a diagonal Gaussian with the same mean and variance as the ”elite” action sequences,
and use this as our action sampling distribution for the next iteration. After repeating this process M times,
we take the final mean of the Gaussian as the optimized action sequence. See Section 3.3 in this paper for
more details.

Additionally, since our model is imperfect and things will never go perfectly according to plan, we adopt a
model predictive control (MPC) approach, where at every time step we perform random-shooting or CEM to
select the best H-step action sequence, but then we execute only the first action from that sequence before
replanning again at the next time step using updated state information. This reduces the effect of compounding
errors when using our approximate dynamics model to plan too far into the future.

\subsection{On-Policy Data Collection}
Although MBRL is in theory off-policy—meaning it can learn from any data—in practice it will perform
poorly if you don’t have on-policy data. In other words, if a model is trained on only randomly-collected data,
it will (in most cases) be insufficient to describe the parts of the state space that we may actually care about.
We can therefore use on-policy data collection in an iterative algorithm to improve overall task performance.
This is summarized as follows:

\begin{algorithm}[h]
Model-Based RL with On-Policy Data
\label{alg:mbrl}
\begin{algorithmic}[1]
\State Use a base policy $\pi_0$ (e.g., random) to collect an initial dataset $\D=\{(\st,\at,\stp)\}$
\While{not done}
    \State Train $\f$ on $\D$ using \eqref{eq:loss-raw} or \eqref{eq:loss-norm}
    \State $\st \gets$ current agent state
    \For{rollout $m=1$ to $M$}
        \For{$t=0$ to $T$}
            \State $A^\star \gets \pi_{\mathrm{MPC}}(\st)$ \Comment{solve \eqref{eq:shooting} by random shooting or CEM}
            \State $\at \gets$ first action of $A^\star$
            \State Execute $\at$, observe $\stp$, set $\st\gets \stp$
            \State Append $(\st,\at,\stp)$ to $\D$
        \EndFor
    \EndFor
\EndWhile
\end{algorithmic}
\end{algorithm}

\subsection{Ensembles}
A simple and effective way to improve predictions is to use an ensemble of models. The idea is simple: rather
than training one network $f_\theta$ to make predictions, we’ll train $N$ independently initialized networks $\{f_{\theta_n}\}_{n=1}^N$.
At test time, for each candidate action sequence, we’ll generate $N$ independent rollouts and average the rewards
of these rollouts to choose the best action sequence.

\section{Code}
You will implement the MBRL algorithm described in the previous section.
\subsection{Overview}
Obtain the code from $https://github.com/berkeleydeeprlcourse/homework_fall2023/tree/$
master/hw4. You will be implementing a model-based RL agent in cs285/agents/model based agent.py.
Make sure to also read the following files:
• cs285/env configs/mpc config.py: generates all of the configuration for the model-based agent.

\newpage
\section{Problem1}
What you will implement:

Collect a large dataset by executing random actions. Train a neural network dynamics model on this fixed
dataset. The implementation that you will do here will be for training the dynamics model.

What code files to fill in:

1. \texttt{cs285/agents/model\_based\_agent.py: up to (and including) update\_statistics.}

2. \texttt{cs285/scripts/run\_hw4.py: everything except for collect\_mbpo\_rollout at the top of the
file.}

What command to run:

\texttt{python cs285/scripts/run\_hw4.py -cfg experiments/mpc/halfcheetah\_0\_iter.yaml}

This config will only run the first iteration without actually evaluating the policy, meaning it will only train the
ensemble of dynamics models. The code will produce plots inside your logdir that illustrate the full learning
curve of the dynamics models. For the first command, the loss should go below 0.2 by iteration 500.

Modify \texttt{experiments/mpc/halfcheetah\_0\_iter.yaml} to change some hyperparameters. Try at least
two other configurations of hyperparameters that affect dynamics model training (e.g., number of layers,
hidden size, or learning rate).

What to submit: For this question, submit the learning curve for 3 runs total: the initial run with provided
hyperparameters as well as 2 of your own.

Note that for these learning curves, we intend for you to just copy the png images produced by the code.

\textbf{Solution}
\begin{verbatim}
export PYTHONPATH=/mnt/f/2025FirstSemester/DeepRL/LAB/homework_fall2023
python hw4/cs285/scripts/run_hw4.py -cfg hw4/experiments/mpc/halfcheetah_0_iter.yaml
python hw4/cs285/scripts/run_hw4.py -cfg hw4/experiments/mpc/halfcheetah_0_iter_lr1.yaml
\end{verbatim}
\begin{figure}[H]
    \centering
    \includegraphics[width=0.48\linewidth]{lr_halfcheetah_0_iter.png}
    \caption{Learning curve for the dynamics model on HalfCheetah with increased learning rate.}
    \label{fig:lr_halfcheetah_0_iter}
\end{figure}
明显看到lr提高10倍后,dynamic model的loss下降更快,且最终loss更低。

\newpage
\section{Problem2}
What will you implement:

Action selection using your learned dynamics model and a given reward function.

What code files to fill in:

1. \texttt{cs285/agents/model\_based\_agent.py}: the rest of the file, except for the CEM strategy in
get\_action.

What commands to run:

\texttt{python cs285/scripts/run\_hw4.py -cfg experiments/mpc/obstacles\_1\_iter.yaml}

Recall the overall flow of our training loop. We first collect data with our policy (which starts as random), we
train our model on that collected data, we evaluating the resulting MPC policy (which now uses the trained
model), and repeat. To verify that your MPC is indeed doing reasonable action selection, run one iteration of
this process using the command above. This will evaluate your MPC policy, but not use it to collect data for
future iterations. Look at eval\_return, which should be greater than -70 after one iteration.

What to submit:

Submit this run as part of your run logs, and report your eval\_return.

\textbf{Solution}
\begin{verbatim}
export PYTHONPATH=/mnt/f/2025FirstSemester/DeepRL/LAB/homework_fall2023
python hw4/cs285/scripts/run_hw4.py -cfg hw4/experiments/mpc/obstacles_1_iter.yaml
\end{verbatim}
\begin{figure}[H]
    \centering
    \includegraphics[width=0.48\linewidth]{obstacle_1_iter_eval_return.png}
    \caption{Learning curve for the dynamics model on Obstacles.}
    \label{fig:lr_obstacle_1_iter}
\end{figure}

\newpage
\section{Problem3}
What will you implement:

MBRL algorithm with on-policy data collection and iterative model training.

What code files to fill in:

None. You should already have done everything that you need, because \texttt{run\_hw4.py} already aggregates
your collected data into a replay buffer. Thus, iterative training means to just train on our growing replay
buffer while collecting new data at each iteration using the most newly trained model.

What commands to run:

\texttt{python cs285/scripts/run\_hw4.py -cfg experiments/mpc/obstacles\_multi\_iter.yaml}

\texttt{python cs285/scripts/run\_hw4.py -cfg experiments/mpc/reacher\_multi\_iter.yaml}

\texttt{python cs285/scripts/run\_hw4.py -cfg experiments/mpc/halfcheetah\_multi\_iter.yaml}

You should expect rewards of around -25 to -20 for the obstacles env, rewards of around -300 to -250 for the
reacher env, and rewards of around 250-350 for the cheetah env.

What to submit:

Submit these runs as part of your run logs, and include the return plots in your pdf.

\textbf{Solution}
\begin{verbatim}
export PYTHONPATH=/mnt/f/2025FirstSemester/DeepRL/LAB/homework_fall2023
python hw4/cs285/scripts/run_hw4.py -cfg hw4/experiments/mpc/obstacles_multi_iter.yaml
python hw4/cs285/scripts/run_hw4.py -cfg hw4/experiments/mpc/reacher_multi_iter.yaml
python hw4/cs285/scripts/run_hw4.py -cfg hw4/experiments/mpc/halfcheetah_multi_iter.yaml
\end{verbatim}

\newpage
\section{Problem4}
What will you implement:

You will compare the performance of your MBRL algorithm as a function of three hyperparameters: the
number of models in your ensemble, the number of random action sequences considered during each action
selection, and the MPC planning horizon.

What code files to fill in:

None.

What commands to run:

python \texttt{cs285/scripts/run\_hw4.py -cfg experiments/mpc/reacher\_ablation.yaml}

Modify (or make copies of) the YAML file to ablate each of the hyperparameters. For each hyperparameter,
do at least 1 run with it increased and 1 with it decreased from the default (so 7 runs total). Make sure to
keep the other hyperparameters the same when studying the effect of one of them.

What to submit:

1. Submit these runs as part of your run logs.

2. Include the following plots (as well as captions that describe your observed trends) of the following:

• effect of ensemble size

• effect of the number of candidate action sequences

• effect of planning horizon

Be sure to include titles and legends on all of your plots.

\textbf{Solution}
\begin{verbatim}
export PYTHONPATH=/mnt/f/2025FirstSemester/DeepRL/LAB/homework_fall2023
python hw4/cs285/scripts/run_hw4.py -cfg hw4/experiments/mpc/reacher_ablation.yaml
python hw4/cs285/scripts/run_hw4.py -cfg hw4/experiments/mpc/reacher_ablation_ensemble_1.yaml &&\
python hw4/cs285/scripts/run_hw4.py -cfg hw4/experiments/mpc/reacher_ablation_ensemble_6.yaml &&\
python hw4/cs285/scripts/run_hw4.py -cfg hw4/experiments/mpc/reacher_ablation_action_sequence_200.yaml
python hw4/cs285/scripts/run_hw4.py -cfg hw4/experiments/mpc/reacher_ablation_action_sequence_4000.yaml &&\
python hw4/cs285/scripts/run_hw4.py -cfg hw4/experiments/mpc/reacher_ablation_horizon_5.yaml &&\
python hw4/cs285/scripts/run_hw4.py -cfg hw4/experiments/mpc/reacher_ablation_horizon_20.yaml
\end{verbatim}


\begin{figure}[H]
    \centering
    \begin{subfigure}[b]{0.48\linewidth}
        \centering
        \includegraphics[width=\linewidth]{reacher_ensemble_eval_return.png}
        \caption{Eval return Curve}
        \label{fig:reacher_ensemble_eval_return}
    \end{subfigure}
    \hfill
    \begin{subfigure}[b]{0.48\linewidth}
        \centering
        \includegraphics[width=\linewidth]{reacher_ensemble_dynamics_loss.png}
        \caption{Dynamics Loss Curve}
        \label{fig:reacher_ensemble_dynamics_loss}
    \end{subfigure}
    \caption{Learning Curves for MPC ensemble size on Reacher.}
    \label{fig:reacher_ensemble_curves}
\end{figure}
ensem=3 > ensem=6 > ensem=1 

适度增大模型集成规模能减轻单模型偏差与不确定性对 MPC 评估的影响,从而提高早期样本效率与最终性能;但收益呈现边际递减,且计算成本随之线性上升。

overly large ensembles can lead to more conservative planning, slower per-model learning, and slightly lower asymptotic performance.

\begin{figure}[H]
    \centering
    \begin{subfigure}[b]{0.48\linewidth}
        \centering
        \includegraphics[width=\linewidth]{reacher_action_sequence_eval_return.png}
        \caption{Eval return Curve}
        \label{fig:reacher_action_sequence_eval_return}
    \end{subfigure}
    \hfill
    \begin{subfigure}[b]{0.48\linewidth}
        \centering
        \includegraphics[width=\linewidth]{reacher_action_sequence_dynamics_loss.png}
        \caption{Dynamics Loss Curve}
        \label{fig:reacher_action_sequence_dynamics_loss}
    \end{subfigure}
    \caption{Learning Curves for MPC action sequence length on Reacher.}
    \label{fig:reacher_action_sequence_curves}
\end{figure}

随机动作序列数量越多,初期性能提升越快、学习结果更稳定。但当序列数量超过约 1000 后,性能增益趋于饱和,进一步增加只会提高计算开销。因此,mpc num\_action\_sequences=1000 在 Reacher 环境中提供了最佳的性能与效率权衡

\begin{figure}[H]
    \centering
    \begin{subfigure}[b]{0.48\linewidth}
        \centering
        \includegraphics[width=\linewidth]{reacher_horizon_eval_return.png}
        \caption{Eval return Curve}
        \label{fig:reacher_horizon_eval_return}
    \end{subfigure}
    \hfill
    \begin{subfigure}[b]{0.48\linewidth}
        \centering
        \includegraphics[width=\linewidth]{reacher_horizon_dynamics_loss.png}
        \caption{Dynamics Loss Curve}
        \label{fig:reacher_horizon_dynamics_loss}
    \end{subfigure}
    \caption{Learning Curves for MPC planning horizon on Reacher.}
    \label{fig:reacher_horizon_curves}
\end{figure}

适度增加规划 horizon 能提升早期样本效率与最终性能,但过长的 horizon 会因模型误差积累而导致性能下降。因此,mpc horizon=10 在 Reacher 环境中提供了最佳的性能与效率权衡。

\newpage
\section{Problem5}
What will you implement:

You will compare the performance of your MBRL algorithm with action selecting performed by random-
shooting (what you have done up to this point) and CEM.

Because CEM can be much slower than random-shooting, we will only run MBRL for 5 iterations for this
problem. We will try two hyperparameter settings for CEM and compare their performance to random-
shooting.

What code files to fill in:

1. \texttt{cs285/agents/model\_based\_agent.py}: the CEM action selection strategy.

What commands to run:

\texttt{python cs285/scripts/run\_hw4.py -cfg experiments/mpc/halfcheetah\_cem.yaml}

You should expect rewards around 800 or higher when using CEM on the cheetah env. Try a cem\_iterations
value of both 2 and 4, and compare results.

What to submit:

1) Submit these runs as part of your run logs.

2) Include a plot comparing random shooting (from Problem 3) with CEM, as well as captions that describe
how CEM affects results for different numbers of sampling iterations (2 vs. 4).

\textbf{Solution}
\begin{verbatim}
export PYTHONPATH=/mnt/f/2025FirstSemester/DeepRL/LAB/homework_fall2023
python hw4/cs285/scripts/run_hw4.py -cfg hw4/experiments/mpc/halfcheetah_cem.yaml
python hw4/cs285/scripts/run_hw4.py -cfg hw4/experiments/mpc/halfcheetah_cem_2.yaml
\end{verbatim}
\begin{figure}[H]
    \centering
    \begin{subfigure}[b]{0.48\linewidth}
        \centering
        \includegraphics[width=\linewidth]{halfcheetah_cem_eval_return.png}
        \caption{Eval return Curve}
        \label{fig:halfcheetah_cem_eval_return}
    \end{subfigure}
    \hfill
    \begin{subfigure}[b]{0.48\linewidth}
        \centering
        \includegraphics[width=\linewidth]{halfcheetah_cem_dynamic_loss.png}
        \caption{Dynamics Loss Curve}
        \label{fig:halfcheetah_cem_dynamics_loss}
    \end{subfigure}
    \caption{Learning Curves for MPC cem on HalfCheetah.}
    \label{fig:halfcheetah_cem_curves}
\end{figure}
CEM 明显优于随机采样,且迭代次数越多,性能越好,但计算开销也更大。cem\_iterations=4 在 HalfCheetah 环境中提供了最佳的性能与效率权衡。

\newpage
\section{Problem6}
What will you implement:

In this homework you will also be implementing a variant of MBPO. Another way of leveraging the learned
model is through generating additional samples to train the policy and value functions. Since RL often
requires many environment interaction samples, which can be costly, we can use our learned model to generate
additional samples to improve sample complexity. In MBPO, we build on your SAC implementation from HW3
and use the learned model you implemented in the earlier questions for generating additional samples to train
our SAC agent. We will try three settings:

1. Model-free SAC baseline: no additional rollouts from the learned model.

2. Dyna (technically “dyna-style” - the original Dyna algorithm is a little different): add single-step rollouts
from the model to the replay buffer and incorporate additional gradient steps per real world step.

3. MBPO: add in 10-step rollouts from the model to the replay buffer and incorporate additional gradient
steps per real world step.

What code files to fill in:

1. \texttt{cs285/scripts/run\_hw4.py}: the collect\_mbpo\_rollout function at the top of the file.
What commands to run:

\texttt{python cs285/scripts/run\_hw4.py -cfg experiments/mpc/halfcheetah\_mbpo.yaml --sac\_config\_file experiments/sac/halfcheetah\_clipq.yaml}

Edit experiments/sac/halfcheetah\_clipq.yaml to change the MBPO rollout length. The model-free
SAC baseline corresponds to a rollout length of 0, The Dyna-like algorithm corresponds to a rollout length of
1, and full MBPO corresponds to a rollout length of 10.

You should be able to reach returns around 700 or higher with full MBPO with a rollout length of 10.
What to submit:

1) Submit these 3 runs as part of your run logs.

2) Include a plot to show a comparison between the 3 runs, and explain any trends you see.


\textbf{Solution}
\begin{verbatim}
export PYTHONPATH=/mnt/f/2025FirstSemester/DeepRL/LAB/homework_fall2023
python hw4/cs285/scripts/run_hw4.py -cfg hw4/experiments/mpc/halfcheetah_mbpo.yaml --sac_config_file hw4/experiments/sac/halfcheetah_clipq.yaml
python hw4/cs285/scripts/run_hw4.py -cfg hw4/experiments/mpc/halfcheetah_mbpo_Dyna.yaml --sac_config_file hw4/experiments/sac/halfcheetah_clipq_Dyna.yaml
python hw4/cs285/scripts/run_hw4.py -cfg hw4/experiments/mpc/halfcheetah_mbpo_MBPO.yaml --sac_config_file hw4/experiments/sac/halfcheetah_clipq_MBPO.yaml
\end{verbatim}
\begin{figure}[H]
    \centering
    \begin{subfigure}[b]{0.48\linewidth}
        \centering
        \includegraphics[width=\linewidth]{cheetah_mbpo_eval_return.png}
        \caption{Eval return Curve}
        \label{fig:cheetah_mbpo_eval_return}
    \end{subfigure}
    \hfill
    \begin{subfigure}[b]{0.48\linewidth}
        \centering
        \includegraphics[width=\linewidth]{cheetah_mbpo_eval_std.png}
        \caption{Eval Std Curve}
        \label{fig:cheetah_mbpo_eval_std}
    \end{subfigure}
    \caption{Learning Curves for MPC mbpo on Cheetah.}
    \label{fig:cheetah_mbpo_curves}
\end{figure}
MBPO 显著优于 Dyna-like 和纯 model-free SAC,且收敛更快、更稳定。Dyna-like 也优于纯 model-free SAC,但提升有限,且不如 MBPO 明显。原因在于,MBPO 通过多步模型预测生成更丰富的虚拟样本,能更有效地利用模型信息提升策略学习效率;而 Dyna-like 仅使用单步预测,受限于模型误差,难以充分发挥模型优势。

\end{document}